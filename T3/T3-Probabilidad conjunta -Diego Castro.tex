
\documentclass[12pt,a4paper]{article}
\usepackage[spanish]{babel}
\usepackage[utf8]{inputenc}
\usepackage{amsmath, amssymb}
\usepackage{graphicx}
\usepackage{geometry}
\usepackage{xcolor}
\usepackage{listings}
\usepackage{hyperref}
\geometry{margin=2.5cm}

\title{\textbf{Ejercicios de Probabilidad Conjunta con Python}}
\author{Diego Castro}
\date{\today}

\lstdefinestyle{pythonstyle}{
    language=Python,
    basicstyle=\ttfamily\footnotesize,
    backgroundcolor=\color{gray!5},
    keywordstyle=\color{blue}\bfseries,
    stringstyle=\color{orange!70!black},
    commentstyle=\color{green!40!black}\itshape,
    showstringspaces=false,
    frame=single,
    framerule=0.5pt,
    breaklines=true,
    tabsize=4,
    numbers=left,
    numberstyle=\tiny\color{gray!80},
    xleftmargin=1.5em,
}

\begin{document}

\maketitle

\section*{Descripción General}
En este documento se presentan varios ejercicios de \textbf{probabilidad conjunta}, tanto discretos como continuos, resueltos mediante código en Python.  
Cada ejercicio incluye el cálculo de funciones de probabilidad, marginales, esperanzas, varianzas, covarianza y correlación, además de representaciones gráficas.

---

\section{Ejercicio 1: Selección de Estudiantes}
\textbf{Enunciado:}  
Se seleccionan al azar dos estudiantes en un salón que contiene 3 de sistemas, 2 de electrónica y 3 de industrial.  

\begin{lstlisting}[style=pythonstyle,caption={Probabilidad conjunta discreta de estudiantes seleccionados}]

\end{lstlisting}

Se construye la tabla de la función de probabilidad conjunta (PMF) y se obtienen las marginales \(P(X)\) y \(P(Y)\).  
Posteriormente se calculan las esperanzas, varianzas, covarianza y coeficiente de correlación.

\begin{align*}
E[X] &= \sum_x x P(X=x) \\
E[Y] &= \sum_y y P(Y=y) \\
\text{Cov}(X,Y) &= E[XY] - E[X]E[Y]
\end{align*}

---

\section{Ejercicio 2: Fábrica de Dulces}
\textbf{Enunciado:}  
Una fábrica de dulces distribuye cajas cuya función de densidad está dada por:
\[
f(x,y) = \frac{2}{5}(2x + 3y), \quad 0 < x < 1, \; 0 < y < 1
\]

Se obtiene la probabilidad en la región \(B\):
\[
P(B) = \frac{13}{160}
\]

y se generan los gráficos de \(f(x,y)\), \(f_X(x)\) y \(f_Y(y)\).

---

\section{Ejercicio 3: Función de Probabilidad Discreta}
\[
f(x,y) = \frac{x + y}{30}, \quad x=0,1,2,3; \, y=0,1,2
\]

Se pide determinar:  
(a) \(P(X+Y=4)\), (b) \(P(X>Y)\), (c) \(P(X \ge 2, Y \le 1)\), (d) \(P(X \le 2, Y = 1)\).

---

\section{Ejercicio 4: Casos Mixtos (Continuo y Discreto)}

\subsection*{Ejercicio A (Continuo)}
\[
f(x,y) = \frac{x+y}{3}, \quad 0 \le x \le 2,\; 0 \le y \le 1
\]
\[
f_X(x) = \frac{2x+1}{6}, \qquad f_Y(y) = \frac{2+2y}{3}
\]

\subsection*{Ejercicio B (Discreto)}
\[
P(X=x, Y=y) = \frac{2x + y}{75}, \quad x=0,\dots,4;\; y=0,\dots,2
\]

Se verifican las propiedades de normalización, se calculan esperanzas, varianzas y covarianza de la distribución.

---

\section*{Conclusión}
Estos ejercicios permiten comprender cómo manipular distribuciones conjuntas discretas y continuas, calcular momentos estadísticos, y representar visualmente los resultados mediante Python.  
Se observa además cómo usar bibliotecas como \texttt{fractions}, \texttt{numpy}, \texttt{pandas} y \texttt{matplotlib} para el análisis probabilístico.

\end{document}
